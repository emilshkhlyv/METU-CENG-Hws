\documentclass[12pt]{article}
\usepackage[utf8]{inputenc}
\usepackage{float}
\usepackage{amsmath}


\usepackage[hmargin=3cm,vmargin=6.0cm]{geometry}
\topmargin=-2cm
\addtolength{\textheight}{6.5cm}
\addtolength{\textwidth}{2.0cm}
\setlength{\oddsidemargin}{0.0cm}
\setlength{\evensidemargin}{0.0cm}
\usepackage{indentfirst}
\usepackage{amsfonts}

\begin{document}

\section*{Student Information}

Name : Emil Shikhaliyev\\

ID : 2280386\\


\section*{Answer 1}
\subsection*{a)}
First we should calculate the \textbf{probability mass function} for all values of \textbf{X}:
\begin{equation}
    \begin{split}
        P_X (0) = P\{X = 0\} = \sum_{y} P_{(X, Y)} (0, y) = P(0, 0) + P(0, 2) = \dfrac{1}{12} + \dfrac{2}{12} = \dfrac{3}{12} \\
         P_X (1) = P\{X = 1\} = \sum_{y} P_{(X, Y)} (1, y) = P(1, 0) + P(1, 2) = \dfrac{4}{12} + \dfrac{2}{12} = \dfrac{6}{12} \\
         P_X (2) = P\{X = 2\} = \sum_{y} P_{(X, Y)} (2, y) = P(2, 0) + P(2, 2) = \dfrac{1}{12} + \dfrac{2}{12} = \dfrac{3}{12} \\
    \end{split}
\end{equation}
Calculation of \textbf{expected value} of random variable \textbf{X}:
\begin{equation}
    \begin{split}
        \mu = E(X) &= \sum_{x} xP(x) \\
        & = 0\cdot P(0) + 1 \cdot P(1) + 2 \cdot P(2) \\
        & = 0 \cdot \dfrac{3}{12} + 1 \cdot \dfrac{6}{12} + 2 \cdot \dfrac{3}{12} = \dfrac{6}{12} + \dfrac{6}{12} = 1
    \end{split}
\end{equation}
Calculation of \textbf{variance} of random variable \textbf{X}:
\begin{equation}
    \begin{split}
        \sigma^2 = \text{Var}(x) & = \sum_{x} (x-\mu)^{2} P(x) \\
        & = (0 - 1)^2 \cdot P(0) + (1-1)^2 \cdot P(1) + (2 - 1)^2 \cdot P(2) \\
        & = 1 \cdot \dfrac{3}{12} + 1 \cdot \dfrac{3}{12}  = \dfrac{3}{12} + \dfrac{3}{12} = \dfrac{6}{12} = \dfrac{1}{2}
    \end{split}
\end{equation}
\subsection*{b)}
Let $Z = X + Y$. So, \textbf{minimum} value of $Z$ is $0 + 0 =0$ and \textbf{maximum} value of $Z$ is $2+2 = 4$. Next, \\
\begin{equation}
    \begin{split}
        P_Z (0) &= P\{Z = 0\} \\ 
                &= P\{X = 0, Y = 0\} = \dfrac{1}{12}\\
        P_Z (1) &= P\{Z = 1\} \\
                &= P\{X = 1, Y = 0\} = \dfrac{4}{12}\\
        P_Z (2) &= P\{Z = 2\} \\
                &= P\{X = 0, Y = 2\} + P\{X = 2, Y = 0\} = \dfrac{2}{12} + \dfrac{1}{12} = \dfrac{3}{12}\\
        P_Z (3) &= P\{Z = 3\}\\
                &= P\{X = 1, Y = 2\} = \dfrac{2}{12}\\
        P_Z (4) &= P\{Z = 4\}\\
                &= P\{X = 2, Y = 2\} = \dfrac{2}{12}
    \end{split}
\end{equation}
\subsection*{c)}
First we should calculate the \textbf{probability mass function} for all values of \textbf{Y}:
\begin{equation}
    \begin{split}
        P_Y (0) = P\{Y = 0\} = \sum_x P_{(X,Y)} (x, 0) = P(0, 0) + P(1, 0) + P(2, 0) = \dfrac{1}{12} + \dfrac{4}{12} + \dfrac{1}{12} = \dfrac{6}{12}\\
        P_Y (2) = P\{Y = 2\} = \sum_x P_{(X,Y)} (x, 2) = P(0, 2) + P(1, 2) + P(2, 2) = \dfrac{2}{12} + \dfrac{2}{12} + \dfrac{2}{12} = \dfrac{6}{12}\\
    \end{split}
\end{equation}
After that we should calculate the \textbf{expected value} of random variable Y: \\
\begin{equation}
    \begin{split}
        \mu = E(Y) &= \sum_y yP(Y) \\
        &= 0 \cdot P(0) + 2 \cdot P(2) = 0 \cdot \dfrac{6}{12} + 2 \cdot \dfrac{6}{12} = \dfrac{12}{12} = 1.  
    \end{split}
\end{equation}
The formula of \textbf{covariance} of random variables X and Y is: 
\begin{equation}
    \begin{split}
        \text{Cov}(X, Y) = \textbf{E}(XY) - \textbf{E}(X)\textbf{E}(Y) 
    \end{split}
\end{equation}
\textbf{E}$(X)$ = 1. (obtained from 1.(a))\\
\textbf{E}$(Y)$ = 1.\\
We should calculate \textbf{E}$(XY)$:
\begin{equation}
    \begin{split}
        \textbf{E}(XY) &= \sum _ x \sum _ y xyP(x, y) \\
        &= (0)(0)P(0,0) + (0)(2)P(0,2) + (1)(0)P(1,0) + (1)(2)P(1,2) + (2)(0)P(2,0) + (2)(2)P(2,2)\\
        &= (1)(2)P(1,2) + (2)(2)P(2,2)\\ 
        &= 1 \cdot 2 \cdot \dfrac{2}{12} +  2 \cdot 2 \cdot \dfrac{2}{12} \\
        &= \dfrac{4}{12} + \dfrac{8}{12} = \dfrac{12}{12} = 1.
    \end{split}
\end{equation}
Calculation of \textbf{covariance} of random variables X and Y:
\begin{equation}
    \begin{split}
        \text{Cov}(X,Y) &= \textbf{E}(XY) - \textbf{E}(X)\textbf{E}(Y)\\
                        &= 1 - 1 \cdot 1  = 0
    \end{split}
\end{equation}

\subsection*{d)}
The formula of \textbf{covariance} is:
\begin{equation}
    \begin{split}
        \text{Cov}(X, Y) = \textbf{E}(XY) - \textbf{E}(X)\textbf{E}(Y) 
    \end{split}
\end{equation}
For \textbf{independent} X and Y, \textit{(from (3.5) on page 49 on textbook)}
\begin{equation}
    \begin{split}
        \textbf{E}(XY) = \textbf{E}(X)\textbf{E}(Y)
    \end{split}
\end{equation}
So, 
\begin{equation}
    \begin{split}
        \text{Cov}(X,Y) &= \textbf{E}(XY) - \textbf{E}(X)\textbf{E}(Y)\\
                        &= \textbf{E}(X)\textbf{E}(Y) - \textbf{E}(X)\textbf{E}(Y) = 0.
    \end{split}
\end{equation}
\subsection*{e)}
If random variables $X$ and $Y$ are independent:
\begin{equation}
    \begin{split}
        P_{(X,Y)}(x,y) = P_X (x) P_Y (y)
    \end{split}
\end{equation}
We should try for all values of random variables X and Y:
\begin{equation}
    \begin{split}
        P_X (0) P_Y (0) = \dfrac{3}{12} \cdot \dfrac{6}{12} = \dfrac{1}{4} \cdot \dfrac{1}{2} = \dfrac{1}{8} \neq P_{(X,Y)}(x,y) = \dfrac{1}{12}
    \end{split}
\end{equation} 
So, the random variables X and Y are \textbf{not independent}.
\section*{Answer 2}
\subsection*{a)}
We can solve this question with \textbf{Binomial distribution}.
Binomial probability mass function is:
\begin{equation}
    \begin{split}
        P(x) = P\{X = x\} = \binom{n}{x} p^x q^{n-x}
    \end{split}
\end{equation}
We need to find probability of $P\{X \geq 3\}$, where $X$ is the number of pens out of $12$, which are broken. $X$ has binomial distribution with parameters $n = 12$, $p = 0.2$ and $q = 0.8$. We can obtain $P\{X \geq 3\}$ in this way:
\begin{equation}
    \begin{split}
        P\{X \geq 3\} = 1 - F(2)
    \end{split}
\end{equation}
The value of $F(2) = 0.558$, obtained from \textit{Table A2} from textbook. Let's calculate $F(2)$ by ourselves:
\begin{equation}
    \begin{split}
        F(2) &= \sum_{x = 0}^{2} P(x) \\
        &= \sum_{x = 0}^{2} \binom{n}{x} p^x q^{n-x}\\
        &= \binom{12}{0} (0.2)^{0} (0.8)^{12} + \binom{12}{1} (0.2)^{1} (0.8)^{11} + \binom{12}{2} (0.2)^{2} (0.8)^{10}\\
        &= 1 \cdot 1 \cdot 0.0687 + 12\cdot 0.2 \cdot 0.0858 + 66 \cdot 0.04 \cdot 0.1073 \\
        &= 0.0687 + 0.2059 + 0.2832 \\
        &= 0.5578 \approx 0.558
    \end{split}
\end{equation}
After that we can put this value in equation (15):
\begin{equation}
    \begin{split}
         P\{X \geq 3\} = 1 - F(2) = 1 - 0.558 = 0.442
    \end{split}
\end{equation}
\subsection*{b)}
We can solve this question with \textbf{Negative Binomial distribution}. Negative Binomial probability function is:
\begin{equation}
    \begin{split}
        P(x) = \binom{x-1}{k-1} p^k (1-p)^{x-k}.
    \end{split}
\end{equation}
Where $p$ is probability of success \textit{(being broken)}.
We need to find probability of $P(5)$ with parameters $k = 2$ and $p = 0.2$. So,
\begin{equation}
    \begin{split}
        P(5) &= \binom{4}{1} (0.2)^2 (0.8)^3\\
             &= 4 \cdot 0.04 \cdot 0.512 \\
             &= 0.08192 \approx 0.082
    \end{split}
\end{equation}
\subsection*{c)}
In this question we can use property of \textbf{Negative Binomial distribution}. \textit{(3.12) on page 63 from textbook}
\begin{equation}
    \begin{split}
        E(X) = \dfrac{k}{p} = \dfrac{4}{0.2} = 20
    \end{split}
\end{equation}
\section*{Answer 3}
\subsection*{a)}
In this question: 
\begin{equation}
    \begin{split}
        \mu = E(X) = 4
    \end{split}
\end{equation}
We can calculate $\lambda$:
\begin{equation}
    \begin{split}
        \lambda = \dfrac{1}{\mu} = \dfrac{1}{4} = \dfrac{1}{4}
    \end{split}
\end{equation}
Now we should calculate $P(X\geq2)$:
\begin{equation}
    \begin{split}
        P(X\geq2)= 1 - F(2) = 1 - (1 -e^{-\lambda x}) = e^{-\lambda x} = e^{-\frac{1}{4} \cdot 2} = e^{-\frac{1}{2}} = 0.6065
    \end{split}
\end{equation}
\subsection*{b)}
We can solve this question with \textbf{Poisson distribution}.\\
First we should calculate the value of $\lambda$.
If the average number of calls in 4 hours is $1$. So, the average number of calls in 10 hours is $2.5$. So, 
\begin{equation}
    \begin{split}
        \lambda = 2.5   
    \end{split}
\end{equation}
Now we can find the answer
\begin{equation}
    \begin{split}
        \textbf{P}\{X \leq 3 \} = F_{X}(3) = \sum_{x = 0}^{3} e^{- \lambda} \dfrac{\lambda^x}{x!} = \sum_{x = 0}^{3} e^{- 2.5} \dfrac{\left(2.5\right)^x}{x!} = 0.758
    \end{split}
\end{equation}
The value of $F_X(3)$ obtained from \textit{Table A3 on textbook}.
\subsection*{c)}
Exponential distribution has \textbf{memoryless property}:
\begin{equation}
    \begin{split}
        \textbf{P}\{T>t+x|T>t\} = \textbf{P}\{T>x\}
    \end{split}
\end{equation}
That means we should just consider 6 hours part. After that we can solve this question with \textbf{Poisson distribution}. That's why first we should find the value of $\lambda$. If the average number of calls in 4 hours is $1$. So, the average number of calls in 6 hours is $1.5$. So,
\begin{equation}
    \begin{split}
        \lambda = 1.5
    \end{split}
\end{equation}
We should divide this question into 4 parts.\\
\textbf{1)} In 10 hours period, Bob didn't get any call. We should calculate the probability of getting at most 3 calls in 6 hours period:
\begin{equation}
    \begin{split}
        \textbf{P}\{X \leq 3 \} = F_{X}(3) = \sum_{x = 0}^{3} e^{- \lambda} \dfrac{\lambda^x}{x!} = \sum_{x = 0}^{3} e^{- 1.5} \dfrac{\left(1.5\right)^x}{x!} = 0.934
    \end{split}
\end{equation}
\textbf{2)} In 10 hours period Bob got 1 call. We should calculate the probability of getting at most 2 calls in 6 hours period: 
\begin{equation}
    \begin{split}
        \textbf{P}\{X \leq 2 \} = F_{X}(2) = \sum_{x = 0}^{2} e^{- \lambda} \dfrac{\lambda^x}{x!} = \sum_{x = 0}^{2} e^{- 1.5} \dfrac{\left(1.5\right)^x}{x!} = 0.809
    \end{split}
\end{equation}
\textbf{3)} In 10 hours period Bob got 2 calls. We should calculate the probability of getting at most 1 call in 6 hours period: 
\begin{equation}
    \begin{split}
        \textbf{P}\{X \leq 1 \} = F_{X}(1) = \sum_{x = 0}^{1} e^{- \lambda} \dfrac{\lambda^x}{x!} = \sum_{x = 0}^{1} e^{- 1.5} \dfrac{\left(1.5\right)^x}{x!} = 0.558
    \end{split}
\end{equation}
\textbf{4)} In 10 hours period Bob got 3 calls. We should calculate the probability of not getting a call in 6 hours period: 
\begin{equation}
    \begin{split}
        \textbf{P}\{X \leq 0 \} = F_{X}(0) = e^{- 1.5} = 0.223
    \end{split}
\end{equation}
All values obtained from \textit{Table A3. Poisson distribution from textbook}.
\end{document}
