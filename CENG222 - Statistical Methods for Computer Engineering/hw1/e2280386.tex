\documentclass[12pt]{article}
\usepackage[utf8]{inputenc}
\usepackage{float}
\usepackage{amsmath}


\usepackage[hmargin=3cm,vmargin=6.0cm]{geometry}
\topmargin=-2cm
\addtolength{\textheight}{6.5cm}
\addtolength{\textwidth}{2.0cm}
\setlength{\oddsidemargin}{0.0cm}
\setlength{\evensidemargin}{0.0cm}
\usepackage{indentfirst}
\usepackage{amsfonts}

\begin{document}

\section*{Student Information}

Name : Emil Shikhaliyev\\

ID : 2280386\\


\section*{Answer 1}
\subsection*{a)}
    $E_1 = $ $\{$ picking a green ball from box X $\}$ 
    \begin{center}
        $P \{ E_1 \} = \dfrac{2}{6} = \dfrac{1}{3}$   
    \end{center}
\subsection*{b)}
    $R = \{$ Picking red ball $\}$ \\
    $R_1 = \{$ Picking red ball from box X when box X is chosen $\}$ \\
    $R_2 = \{$ Picking red ball from box Y when box Y is chosen $\}$ \\
    $R_3 = \{$ Choosing box X $\}$ \\
    $R_4 = \{$ Choosing box Y $\}$ 
    \begin{center}
        $P\{R\} = P\{R_1\}P\{R_3\} + P\{R_2\}P\{R_4\} = \dfrac{1}{3} \cdot \dfrac{4}{10} + \dfrac{1}{5} \cdot \dfrac{6}{10} = \dfrac{38}{150} = \dfrac{19}{75}$
    \end{center}
\subsection*{c)}
$B = \{$ Picking blue ball $\}$ \\
$Y_1 = \{$ Choosing box Y $\}$ \\
$\overline{Y_1} = \{$ Not choosing box Y $\}$ 
\begin{center}
    $P\{Y_1|B\} = \dfrac{P\{ Y_1 \cap B\}}{P\{B\}} = \dfrac{P\{ Y_1 \cap B\}}{P\{B|Y_1\}P\{Y_1\} + P\{B|\overline{Y_1}\}P\{\overline{Y_1}\}} = \dfrac{\dfrac{4}{10} \cdot \dfrac{6}{10}}{\dfrac{4}{10}\cdot\dfrac{6}{10} + \dfrac{1}{3}\cdot\dfrac{4}{10}} = \dfrac{ 0.24}{\dfrac{112}{300}} =$ \\ $= \dfrac{24}{100}\cdot \dfrac{300}{112} = \dfrac{9}{14}$
\end{center}
\section*{Answer 2}
\subsection*{a)}
The statement, "$A$ and $B$ are $\bold{mutually}$ $\bold{exclusive}$ if and only if $\overline{A}$ and $\overline{B}$ are exhaustive" implies that "If $A$ and $B$ are mutually exclusive, so they are exhaustive." and "If $\overline{A}$ and $\overline{B}$ are exhaustive, so they are mutually exclusive."\\

Let's assume $\overline{A}$ and $\overline{B}$ are exhaustive. So, $\overline{A} \cup \overline{B} = \Omega $.\\
If we take the complement of the both sides of this equation, the equation stays same.  
\begin{center}
    $\overline{\overline{A} \cup \overline{B}} = \overline{\Omega}$.
\end{center}
According to the De Morgan's laws
\begin{center}
    $\overline{\overline{A} \cup \overline{B}} = A \cap B$
\end{center}
The complement of $\Omega$ is $\emptyset$.\\
\begin{center}
    $P\{\Omega\} = 1$\\
$P\{\overline{\Omega}\} = 1 - P\{\Omega\} = 0 = P\{\emptyset\}$.\\
$\overline{\Omega} = \emptyset$.
\end{center}
 So, 
\begin{center}
    $A \cap B = \emptyset$
\end{center} 
Now let's assume $A$ and $B$ are mutually exclusive: $A \cap B = \emptyset$.\\
Take complement of both sides: \\
\begin{center}
    $\overline{A \cap B} = \overline{\emptyset}$
\end{center}
Again according to De Morgan's laws:
\begin{center}
    $\overline{A \cap B} = \overline{A} \cup \overline{B}$
\end{center}
The complement of $\emptyset$ is $\Omega$.
\begin{center}
    $P\{\emptyset\} = 0$\\
    $P\{\overline{\emptyset\}} = 1 - P\{\emptyset\} = 1 - 0 = 1 = P\{\Omega\}$\\
    $\overline{\emptyset} = \Omega$ 
\end{center}
So,
\begin{center}
    $\overline{A} \cup \overline{B} = \Omega$
\end{center}
These results implies that:
\begin{center}
    $A \cap B = \emptyset \Longleftrightarrow \overline{A} \cup \overline{B} = \Omega$\
\end{center}
So, the statement, "A and B are $\bold{mutually}$ $\bold{exclusive}$ if and only if $\overline{A}$ and $\overline{B}$ are exhaustive." has been proven.
\subsection*{b)}
The statement, "$A$, $B$ and $C$ are $\bold{mutually}$ $\bold{exclusive}$ if and only if $\overline{A}$, $\overline{B}$ and $\overline{C}$ are exhaustive" implies that "If $A$, $B$ and $C$ are mutually exclusive, so they are exhaustive." and "If $\overline{A}$, $\overline{B}$ and $\overline{C}$ are exhaustive, so they are mutually exclusive."\\

Let's assume $\overline{A}$, $\overline{B}$ and $\overline{C}$ are exhaustive: $\overline{A} \cup \overline{B} \cup \overline{C} = \Omega$\\
If $A$, $B$ and $C$ sets are mutually exclusive, so the following equations should hold for those sets:
\begin{center}
    $A \cap B = \emptyset$\\
    $A \cap C = \emptyset$\\
    $B \cap C = \emptyset$\\
\end{center}
Let's try these for the following sets $A_1$ ,$B_1$ and $C_1$:
\begin{center}
    $A_1 = \{ 2, 3\}$, $B_1 = \{4, 5, 6\}$ and $C_1 = \{4\}$
\end{center}
Our $\Omega$ is $\{2, 3, 4, 5, 6\}$.\\
\begin{center}
    $\overline{A_1} \cup \overline{B_1} \cup \overline{C_1} = \{2, 3, 4, 5, 6\}$
\end{center}
So, $\overline{A_1}$, $\overline{B_1}$ and $\overline{C_1}$ are exhaustive.\\
\begin{center}
    $A_1 \cap B_1 = \emptyset$\\
    $A_1 \cap C_1 = \{4\}$\\
    $B_1 \cap C_1 = \{4\}$
\end{center}
So, $A_1$, $B_1$ and $C_1$ are not mutually exclusive.\\
As a result, the statement, "$A$, $B$ and $C$ are $\bold{mutually}$ $\bold{exclusive}$ if and only if $\overline{A}$, $\overline{B}$ and $\overline{C}$ are exhaustive." has been disproven.


\section*{Answer 3}
\subsection*{a)}
$E_1 = \{$ having exactly 2 successful dice $\}$ 
\begin{center}
    $P\{E_1\} = \binom{5}{2} \cdot (\frac{1}{3})^2 \cdot (\frac{2}{3})^3 = 10 \cdot \dfrac{1}{9} \cdot \dfrac{8}{27} = \dfrac{80}{243}$
\end{center}
\subsection*{b)}
$E = \{$ having at least 2 successful dice $\}$\\
$E_2 = \{$ having exactly 2 successful dice $\}$\\
$E_3 = \{$ having exactly 3 successful dice $\}$\\
$E_4 = \{$ having exactly 4 successful dice $\}$\\
$E_5 = \{$ having exactly 5 successful dice $\}$ \\ \\
    $P\{E\} = P\{E_2\} + P\{E_3\} +P\{E_4\} +P\{E_5\} =  
\binom{5}{2} (\dfrac{1}{3})^2 (\dfrac{2}{3})^3 + \binom{5}{3} (\dfrac{1}{3})^3 (\dfrac{2}{3})^2 + \binom{5}{4} (\dfrac{1}{3})^4 (\dfrac{2}{3})^1 + \binom{5}{5} (\dfrac{1}{3})^5 (\dfrac{2}{3})^0 =$ \\ \\
$ 10 \cdot \dfrac{1}{9} \cdot \dfrac{8}{27} + 10\cdot \dfrac{1}{27} \cdot \dfrac{4}{9} + 5 \cdot \dfrac{1}{81} \cdot \dfrac{2}{3} + \dfrac{1}{243} = \dfrac{80}{243} + \dfrac{40}{243} + \dfrac{10}{243} + \dfrac{1}{243} = \dfrac{131}{243}$ 

\section*{Answer 4}
\subsection*{a)}
$P_{(A, C)}(1, 0) =
P\{A = 1, C = 0\} = \Sigma_{b} P_{A, B, C} (1, b , 0) = P\{A = 1, B =0, C =0\} + $ \\ $P\{A=1, B=1 ,C=0\} = 0.06 + 0.09 = 0.15$
\subsection*{b)}
$P_{B}(1) = P\{B = 1\} = \Sigma_{a} \Sigma_{c} P_{(A, B, C)} (a, 1, c) =P\{A = 0, B=1, C=0\} + P\{A=0, B=1, C=1\}+ P\{A=1, B=1, C=0\}+ P\{A=1, B=1, C=1\} = 0.21 + 0.02 + 0.09+ 0.08 = 0.4$
\subsection*{c)}
$P_{A}(1) = P\{A = 1\} =\Sigma_{b} \Sigma_{c} P_{(A, B, C)} (1, b, c) = P\{A = 1, B =0, C=0\} + P\{A=1, B=0, C=1\} + P\{A =1, B=1, C=0\} + P\{A=1, B=1, C=1\} = 0.06+ 0.32+ 0.09 + 0.08 = 0.55$ \\

$P_{B}(1) =P\{B = 1\} = \Sigma_{a} \Sigma_{c} P_{(A, B, C)} (a, 1, c) = P\{A = 0, B=1, C=0\} + P\{A=0, B=1, C=1\}+ P\{A=1, B=1, C=0\}+ P\{A=1, B=1, C=1\} = 0.21 + 0.02 + 0.09+ 0.08 = 0.4$ \\

$P_{A, B}(1, 1) = P\{A = 1, B = 1\} = \Sigma_{c}P_{A, B, C} (1, 1, c) = P\{A = 1,B=1,C=0\} + P\{A=1,B=1,C=1\} = 0.09 + 0.08 = 0.17$\\

If A and B are independent, \begin{center}
    $P\{A = 1, B = 1\}$ = $P\{A = 1\} \cdot P\{B=1\}$.
\end{center} But, $P\{A = 1, B = 1\} = 0.17 \neq P\{A= 1\}P\{B=1\} = 0.4 \cdot 0.55 = 0.22$. So A and B are not independent.
\subsection*{d)}
If $A$ and $B$ are conditionally independent, so then the formula \begin{center}
    $P\{A , B | C = 1\} = P\{A|C = 1\} \cdot P\{B|C = 1\}$
\end{center} should hold.\\ \\

First let's calculate \\
$P_{C}(1) = P\{C = 1\} = \Sigma_{a} \Sigma_{b} P_{(A, B, C)} (a, b, 1) = P\{A = 0, B = 0, C = 1\} + P\{A = 0, B = 1,C=1\} + P\{A=1, B=0, C=1\} + P\{A= 1, B =1, C = 1\} = 0.08+0.02+0.32+0.08 = 0.5$ \\ \\
$P_{A, B}(0, 0| C = 1) = P\{A = 0 , B = 0 | C = 1\} = \dfrac{P\{A , B , C\}}{P\{C\}} = \dfrac{0.08}{0.5} = 0.16$ \\
$P\{A = 0 | C = 1\}P\{B = 0 | C = 1\} = \dfrac{P\{A , C\}}{P\{C\}} \cdot \dfrac{P\{B , C \}}{P\{C\}} = \dfrac{0.1}{0.5}\cdot\dfrac{0.4}{0.5} = 0.2 \cdot 0.8 = 0.16$ \\ 
\begin{center}
    $P_{A, B}(0, 0| C = 1) = P\{A = 0 | C = 1\}P\{B = 0 | C = 1\}$ \\ 
\end{center} 
$P_{A, B}(0, 1| C = 1) = P\{A = 0 , B = 1 | C = 1\} = \dfrac{P\{A , B , C\}}{P\{C\}} = \dfrac{0.02}{0.5} = 0.04$\\ \\
$P\{A = 0 , C = 1\}P\{B = 1 , C = 1\} = \dfrac{P\{A , C\}}{P\{C\}} \cdot \dfrac{P\{B , C \}}{P\{C\}} = \dfrac{0.1}{0.5}\cdot\dfrac{0.1}{0.5} = 0.2 \cdot 0.2 = 0.04$ \\ 
\begin{center}
    $P_{A, B}(0, 1| C = 1) = P\{A = 0 | C = 1\}P\{B = 1 | C = 1\}$ \\ 
\end{center}
$P_{A, B}(1, 0| C = 1) = P\{A = 1 , B = 0 | C = 1\} = \dfrac{P\{A , B , C\}}{P\{C\}} = \dfrac{0.32}{0.5} = 0.64$ \\
$P\{A = 1 , C = 1\}P\{B = 0 , C = 1\} = \dfrac{P\{A , C\}}{P\{C\}} \cdot \dfrac{P\{B , C \}}{P\{C\}} = \dfrac{0.4}{0.5}\cdot\dfrac{0.4}{0.5} = 0.8 \cdot 0.8 = 0.64$ \\ 
\begin{center}
    $P_{A, B}(1, 0| C = 1) = P\{A = 1 | C = 1\}P\{B = 0 | C = 1\}$ \\ 
\end{center} 
$P_{A, B}(1, 1| C = 1) = P\{A = 1 , B = 1 | C = 1\} = \dfrac{P\{A , B , C\}}{P\{C\}} = \dfrac{0.08}{0.5} = 0.16$ \\
$P\{A = 1 , C = 1\}P\{B = 1 , C = 1\} = \dfrac{P\{A , C\}}{P\{C\}} \cdot \dfrac{P\{B , C \}}{P\{C\}} = \dfrac{0.4}{0.5}\cdot\dfrac{0.1}{0.5} = 0.8 \cdot 0.2 = 0.16$ \\ 
\begin{center}
    $P_{A, B}(1, 1| C = 1) = P\{A = 1 | C = 1\}P\{B = 1 | C = 1\}$ \\ 
\end{center}

The equation holds for all the values of $A$ and $B$ when $C = 1$. So $A$ and $B$ are conditionally independent for the $C = 1$.
\end{document}

